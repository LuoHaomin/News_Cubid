\documentclass[12pt,a4paper]{article}
\usepackage[UTF8]{ctex}
\usepackage{graphicx}
\usepackage{hyperref}
\usepackage{listings}
\usepackage{xcolor}
\usepackage{booktabs}
\usepackage{multirow}
\usepackage{amsmath}
\usepackage{geometry}
\usepackage{fancyhdr}
\usepackage{titlesec}
\usepackage{enumitem}
\usepackage{setspace}
\usepackage{float}
\usepackage{caption}
\usepackage{subcaption}

% 页面设置
\geometry{a4paper,left=2.5cm,right=2.5cm,top=2.5cm,bottom=2.5cm}
\pagestyle{fancy}
\fancyhf{}
\fancyhead[C]{实验报告}
\fancyfoot[C]{\thepage}
\renewcommand{\headrulewidth}{0.4pt}
\renewcommand{\footrulewidth}{0pt}

% 标题格式
\titleformat{\section}{\Large\bfseries}{\thesection}{1em}{}
\titleformat{\subsection}{\large\bfseries}{\thesubsection}{1em}{}
\titleformat{\subsubsection}{\normalsize\bfseries}{\thesubsubsection}{1em}{}

% 列表格式
\setlist[itemize]{leftmargin=*,itemsep=0.2em,parsep=0.1em}
\setlist[enumerate]{leftmargin=*,itemsep=0.2em,parsep=0.1em}

% 行距
\onehalfspacing

% 代码格式
\lstset{
    basicstyle=\small\ttfamily,
    keywordstyle=\color{blue},
    commentstyle=\color{gray},
    stringstyle=\color{red},
    breaklines=true,
    frame=single,
    numbers=left,
    numberstyle=\tiny\color{gray}
}

% 目录设置
\hypersetup{
colorlinks=true,
linkcolor=black,
citecolor=black,
urlcolor=black,
pdfborder={0 0 0}
}

\title{数据科学导论——新闻推荐\\实验报告}
\author{李丁 PB23111595 \quad 罗浩民 PB23111599}
\date{\today}

\begin{document}

\maketitle
\tableofcontents
\newpage
\section{问题定义}

本赛题是一个新闻推荐任务,目标是预测用户未来点击的新闻文章。数据集包含30万用户、近300万次点击记录和36万多篇不同的新闻文章。任务要求为每个用户预测Top-5最可能点击的文章,评估指标为MRR(Mean Reciprocal Rank)。

\textbf{数据特点:}
\begin{itemize}
\item 训练集:20万用户的点击日志
\item 测试集A:5万用户的点击日志
\item 测试集B:5万用户的点击日志
\item 数据包含用户点击行为、文章信息、文章embedding向量等
\end{itemize}

\textbf{挑战:}
\begin{enumerate}
\item 数据稀疏性:用户-文章交互矩阵非常稀疏
\item 类别不平衡:正负样本比例约为1:196
\item 冷启动问题:新用户和新文章的推荐
\item 时效性:需要考虑时间因素对用户兴趣的影响
\end{enumerate}

\section{方案设计}

\subsection{整体流程}

我们采用经典的"召回+排序"两阶段推荐系统架构,通过多次迭代优化逐步提升模型性能,整体流程如下:

\begin{center}
\fbox{\parbox{0.8\textwidth}{\centering
\textbf{数据预处理} $\rightarrow$ \textbf{召回阶段} $\rightarrow$ \textbf{特征工程} $\rightarrow$ \textbf{排序阶段} $\rightarrow$ \textbf{结果生成}
}}
\end{center}

\begin{itemize}
\item \textbf{召回阶段}:从海量文章中快速筛选出可能相关的候选文章(约154篇/用户)
\item \textbf{排序阶段}:对候选文章进行精细排序,预测用户最可能点击的Top-5文章
\end{itemize}

\subsection{迭代优化过程}

我们通过四次迭代逐步优化模型性能,每次迭代都有明确的改进目标:

\begin{table}[H]
\centering
\small
\caption{迭代优化过程总结}
\begin{tabular}{p{2.5cm}p{4cm}p{4cm}p{2cm}}
\toprule
\textbf{版本} & \textbf{召回策略} & \textbf{排序策略与特征} & \textbf{MRR@5} \\
\midrule
第一版 & 仅ItemCF协同过滤 & LightGBM + 基础特征 & $\approx$ 0.28 \\
\midrule
第二版 & ItemCF + Binetwork + W2V融合 & LightGBM + 相似度特征 + 时间特征 & $\approx$ 0.30 \\
\midrule
第三版 & 优化融合策略(Min-Max标准化) & LightGBM + 用户行为特征 + 时间序列特征 & $\approx$ 0.31 \\
\midrule
第四版 & 保持多路召回融合 & LightGBM + LambdaMART + 列表级别特征 & $\approx$ 0.31 \\
\bottomrule
\end{tabular}
\end{table}

\textbf{关键改进点:}
\begin{itemize}
\item \textbf{召回阶段}:从单一方法到多路召回融合,提升召回覆盖率
\item \textbf{特征工程}:从基础特征到30个精心设计的特征,包括相似度、时间、用户行为等
\item \textbf{排序方法}:从Pointwise到Listwise,直接优化排序指标
\item \textbf{工程优化}:向量化计算、交叉验证、样本权重等,提升训练效率和模型性能
\end{itemize}

\section{数据分析}

\subsection{数据概览}

数据集规模庞大,包含30万用户、近300万次点击记录和36万多篇不同的新闻文章。具体数据分布如下:

\begin{table}[H]
\centering
\small
\caption{数据概览}
\begin{tabular}{lc}
\toprule
\textbf{数据项} & \textbf{数量/比例} \\
\midrule
训练集点击记录 & 1,112,623条 \\
测试集点击记录 & 518,010条 \\
验证集用户数 & 50,000 \\
召回后样本数 & 13,900,088(平均每个用户召回约154篇文章) \\
正负样本比例 & 1:196(严重不平衡) \\
\bottomrule
\end{tabular}
\end{table}

\subsection{特征分析}

我们构建了30个特征,主要分为文章基础特征、召回相似度特征、用户行为特征和列表级别特征四大类。下面用表格形式总结主要特征,便于理解:

\subsubsection{文章基础特征}

文章基础特征直接反映文章本身的属性,这些特征简单但有效,能帮助模型理解文章的基本信息。

\begin{table}[H]
\centering
\small
\caption{文章基础特征总结}
\begin{tabular}{p{5cm}p{3cm}p{6cm}}
\toprule
\textbf{特征名称} & \textbf{统计值} & \textbf{业务含义} \\
\midrule
文章字数(words\_count) & 均值200.83,中位数196 & 反映用户阅读习惯,不同用户对文章长度偏好不同 \\
\midrule
文章类别(category\_id) & - & 反映用户对不同类别文章的偏好,不同类别受欢迎程度差异大 \\
\midrule
文章创建时间(created\_at\_ts) & - & 反映文章时效性,新文章通常更受欢迎,但不同用户对时效性偏好不同 \\
\bottomrule
\end{tabular}
\end{table}


\begin{figure}[H]
\centering
\begin{subfigure}{0.48\textwidth}
\centering
\includegraphics[width=\textwidth]{assets/words_count_distribution.png}
\caption{文章字数分布}
\end{subfigure}
\begin{subfigure}{0.48\textwidth}
\centering
\includegraphics[width=\textwidth]{assets/category_distribution.png}
\caption{文章类别分布}
\end{subfigure}
\caption{文章基础特征分布}
\end{figure}

\textbf{关键发现:}从分布图可以看出,文章字数主要集中在100-300字之间,符合新闻文章的特点。正负样本的字数分布相似,说明文章长度需要与其他特征配合使用。不同类别的文章数量差异很大,某些类别的正样本比例明显高于其他类别,表明类别特征具有重要作用。


\begin{figure}[H]
\centering
\includegraphics[width=0.8\textwidth]{assets/article_creation_time_distribution.png}
\caption{文章创建时间分布}
\end{figure}

\begin{figure}[H]
\centering
\begin{subfigure}{0.48\textwidth}
\centering
\includegraphics[width=\textwidth]{assets/words_count_pos_neg_comparison.png}
\caption{文章字数正负样本对比}
\end{subfigure}
\begin{subfigure}{0.48\textwidth}
\centering
\includegraphics[width=\textwidth]{assets/category_pos_neg_comparison.png}
\caption{文章类别正负样本对比}
\end{subfigure}
\caption{文章基础特征在正负样本上的分布对比}
\end{figure}

\subsubsection{召回相似度特征}

召回相似度特征是模型的核心,它们综合了多种召回方法的结果,反映文章与用户兴趣的匹配程度。这些特征的重要性最高,是模型判断用户是否会点击文章的关键依据。

\begin{table}[H]
\centering
\small
\caption{召回相似度特征总结}
\begin{tabular}{p{5cm}p{3cm}p{6cm}}
\toprule
\textbf{特征名称} & \textbf{统计值} & \textbf{业务含义} \\
\midrule
召回相似度分数(sim\_score) & 均值0.1278,中位数0.0442 & \textbf{最重要的特征},综合ItemCF、Binetwork、W2V三种方法,反映文章与用户历史兴趣的相似度。正样本平均相似度0.8901,负样本0.1206,差异明显 \\
\midrule
用户最后点击文章ItemCF相似度 & - & 捕捉用户\textbf{最近兴趣},反映候选文章与用户最近阅读内容的匹配度 \\
\midrule
用户历史点击文章ItemCF加权相似度之和 & - & 反映用户\textbf{整体兴趣},考虑所有历史点击,近期文章权重更高(0.7幂次衰减) \\
\midrule
用户最后点击文章Binetwork相似度 & - & 基于用户-文章网络结构,与ItemCF互补,能发现ItemCF发现不了的关联 \\
\midrule
用户最后点击文章W2V相似度 & - & 捕捉\textbf{语义相似性},发现内容相似但用户群体不同的文章 \\
\midrule
用户最近2篇点击文章W2V相似度之和 & - & 综合反映候选文章与用户最近阅读内容的语义匹配程度 \\
\bottomrule
\end{tabular}
\end{table}

\textbf{关键发现:}相似度分数呈现明显的长尾分布,大部分文章相似度较低,只有少数文章相似度很高。\textbf{正样本的平均相似度分数(0.8901)远高于负样本(0.1206)},说明相似度分数是强信号,验证了召回阶段的有效性。

\begin{figure}[H]
\centering
\begin{subfigure}{0.48\textwidth}
\centering
\includegraphics[width=\textwidth]{assets/sim_score_distribution.png}
\caption{相似度分数分布}
\end{subfigure}
\begin{subfigure}{0.48\textwidth}
\centering
\includegraphics[width=\textwidth]{assets/sim_score_pos_neg_comparison.png}
\caption{正负样本对比}
\end{subfigure}
\caption{召回相似度特征分析}
\end{figure}

\begin{figure}[H]
\centering
\begin{subfigure}{0.32\textwidth}
\centering
\includegraphics[width=\textwidth]{assets/itemcf_sim_distribution.png}
\caption{ItemCF相似度分布}
\end{subfigure}
\begin{subfigure}{0.32\textwidth}
\centering
\includegraphics[width=\textwidth]{assets/binetwork_sim_distribution.png}
\caption{Binetwork相似度分布}
\end{subfigure}
\begin{subfigure}{0.32\textwidth}
\centering
\includegraphics[width=\textwidth]{assets/w2v_sim_distribution.png}
\caption{W2V相似度分布}
\end{subfigure}
\caption{不同相似度特征分布比较}
\end{figure}

\subsubsection{用户行为特征}

用户行为特征反映用户的活跃度、偏好和阅读习惯,帮助我们理解不同用户的行为模式。

\begin{table}[H]
\centering
\small
\caption{用户统计特征}
\begin{tabular}{p{5cm}p{7cm}}
\toprule
\textbf{特征名称} & \textbf{业务含义} \\
\midrule
用户点击文章数量(user\_id\_cnt) & 反映用户活跃程度,活跃用户和沉默用户的点击行为模式不同 \\
\midrule
文章被点击次数(article\_id\_cnt) & 反映文章热度,热门文章更容易被点击,但需要平衡热门度和多样性 \\
\midrule
用户-类别点击次数(user\_id\_category\_id\_cnt) & 反映用户对特定类别的偏好程度,比单纯类别特征更精细 \\
\bottomrule
\end{tabular}
\end{table}

\textbf{时间相关特征:}新闻推荐中时间因素非常重要,我们构建了多个时间特征来捕捉用户的时效性偏好。

\begin{table}[H]
\centering
\small
\caption{时间相关特征}
\begin{tabular}{p{5cm}p{7cm}}
\toprule
\textbf{特征名称} & \textbf{业务含义} \\
\midrule
候选文章创建时间与用户最后点击时间的差值 & 反映文章新鲜度,时间差小说明文章刚发布,时间差大可能已过时 \\
\midrule
候选文章与用户最后点击文章创建时间的差值 & 反映候选文章与用户最近阅读文章的新鲜度对比 \\
\midrule
用户连续两次点击的平均时间间隔 & 反映用户阅读节奏,帮助判断现在推荐文章是否合适 \\
\midrule
用户点击时间与文章创建时间的差值(均值/标准差) & 均值反映用户通常看多新的文章,标准差反映偏好稳定性 \\
\midrule
用户点击时间的小时标准差 & 反映阅读时间规律性,标准差小说明阅读时间集中 \\
\bottomrule
\end{tabular}
\end{table}

\textbf{文章内容特征:}这些特征反映用户对文章内容属性的偏好。

\begin{table}[H]
\centering
\small
\caption{文章内容特征}
\begin{tabular}{p{5cm}p{7cm}}
\toprule
\textbf{特征名称} & \textbf{业务含义} \\
\midrule
用户历史点击文章的平均字数 & 反映用户对文章长度的长期偏好 \\
\midrule
用户最后点击文章的字数 & 反映用户对文章长度的最近偏好,与平均字数配合捕捉长期和短期变化 \\
\midrule
候选文章字数与用户最后点击文章字数的差值 & 反映文章长度变化幅度,帮助判断用户是否能接受长度变化 \\
\bottomrule
\end{tabular}
\end{table}

\textbf{关键发现:}字数差值主要集中在0附近,说明大部分候选文章的长度和用户最近看的文章长度相近,符合用户阅读习惯:用户通常会在相似长度的文章之间切换。

\begin{figure}[H]
\centering
\includegraphics[width=0.8\textwidth]{assets/words_count_diff_distribution.png}
\caption{候选文章与最近点击文章字数差值分布}
\end{figure}

\begin{figure}[H]
\centering
\begin{subfigure}{0.48\textwidth}
\centering
\includegraphics[width=\textwidth]{assets/user_click_count_distribution.png}
\caption{用户点击数量分布}
\end{subfigure}
\begin{subfigure}{0.48\textwidth}
\centering
\includegraphics[width=\textwidth]{assets/user_click_interval_distribution.png}
\caption{用户点击时间间隔}
\end{subfigure}
\caption{用户行为特征分析}
\end{figure}

\subsubsection{标签分布与特征相关性}

\textbf{正负样本分布:}数据分布严重不平衡,正样本(用户点击的文章)只占0.23\%,负样本占99.77\%,比例接近1:200。这种不平衡在推荐系统中很常见,因为召回阶段会给每个用户召回很多候选文章,但用户只会点击其中的一小部分。我们通过设置正样本权重(scale\_pos\_weight=10)来处理这种不平衡,以提升模型对正样本的识别能力。

\begin{table}[H]
\centering
\caption{正负样本分布}
\begin{tabular}{lcc}
\toprule
\textbf{样本类型} & \textbf{数量} & \textbf{比例} \\
\midrule
正样本 & 40,884 & 0.23\% \\
负样本 & 8,009,580 & 99.77\% \\
\midrule
\textbf{总计} & \textbf{8,050,464} & \textbf{100\%} \\
\bottomrule
\end{tabular}
\end{table}

\begin{figure}[H]
\centering
\includegraphics[width=0.6\textwidth]{assets/label_distribution.png}
\caption{正负样本数量分布}
\end{figure}

\textbf{特征相关性分析:}从特征相关性热力图可以看出:

\begin{itemize}
\item \textbf{相似度特征之间相关性较高}:ItemCF、Binetwork、W2V等相似度特征存在一定相关性,但可以互补
\item \textbf{时间特征相关性较低}:时间相关特征与其他特征相关性低,提供了独立信息
\item \textbf{用户行为特征相关性适中}:与相似度特征有一定相关性,但提供了额外信息
\end{itemize}

\begin{figure}[H]
\centering
\includegraphics[width=0.7\textwidth]{assets/feature_correlation_heatmap.png}
\caption{特征相关性热力图}
\end{figure}

\section{模型设计}

\subsection{召回阶段}

我们采用多路召回策略,融合三种召回方法,从不同角度发现用户可能感兴趣的文章:

\begin{table}[H]
\centering
\small
\caption{召回方法对比}
\begin{tabular}{p{3cm}p{5cm}cc}
\toprule
\textbf{召回方法} & \textbf{原理} & \textbf{HitRate@50} & \textbf{MRR@50} \\
\midrule
ItemCF(基于物品的协同过滤) & 计算物品之间的相似度,推荐与用户历史点击物品相似的文章 & 0.70 & 0.238 \\
\midrule
Binetwork(二分网络) & 基于用户-物品二分网络结构,计算物品相似度 & 0.61 & 0.238 \\
\midrule
Word2Vec(词向量) & 将用户点击序列视为句子,训练Word2Vec模型,计算文章相似度 & 0.31 & 0.083 \\
\bottomrule
\end{tabular}
\end{table}

\textbf{召回融合策略:}
\begin{itemize}
\item 对每种召回方法的相似度分数进行Min-Max标准化(将不同方法的分数统一到0-1范围)
\item 加权融合:\texttt{sim\_score = itemcf\_score + binetwork\_score + w2v\_score}
\item \textbf{最终召回指标}:HitRate@50=0.70, MRR@50=0.261(融合后性能优于单一方法)
\end{itemize}

\subsection{排序阶段}

排序学习(Learning to Rank, LTR)是信息检索和推荐系统中的核心问题,目标是根据查询(用户)对文档(物品)的相关性进行排序。根据训练数据的标注方式,排序学习可以分为三类:

\textbf{1. Pointwise方法}
\begin{itemize}
\item \textbf{原理}:将排序问题转化为分类或回归问题,对每个样本独立预测其相关性分数
\item \textbf{优点}:实现简单,可以使用成熟的分类/回归算法
\item \textbf{缺点}:忽略了样本之间的相对顺序关系,无法直接优化排序指标
\item \textbf{代表算法}:LightGBM分类器、逻辑回归、SVM等
\end{itemize}

\textbf{2. Pairwise方法}
\begin{itemize}
\item \textbf{原理}:将排序问题转化为样本对的二分类问题,学习判断两个样本的相对顺序
\item \textbf{优点}:考虑了样本之间的相对关系,更符合排序的本质
\item \textbf{缺点}:样本对数量庞大(O(n²)),训练复杂度高
\item \textbf{代表算法}:RankNet、LambdaRank等
\end{itemize}

\textbf{3. Listwise方法}
\begin{itemize}
\item \textbf{原理}:直接优化整个列表的排序质量,使用排序指标(如NDCG)作为损失函数
\item \textbf{优点}:直接优化排序指标,理论上性能最优
\item \textbf{缺点}:实现复杂,计算开销大
\item \textbf{代表算法}:LambdaMART、ListNet等
\end{itemize}

我们实现并对比了两种排序方法:Pointwise(LightGBM)和Listwise(LambdaMART)。

\subsubsection{LightGBM(Pointwise)}

\textbf{理论原理:}

LightGBM是一个基于梯度提升决策树(GBDT)的框架,我们将其作为Pointwise排序方法使用:

\begin{enumerate}
\item \textbf{Pointwise排序}:将排序问题转化为二分类问题
  \begin{itemize}
  \item 正样本:用户点击的文章(label=1)
  \item 负样本:用户未点击的文章(label=0)
  \item 模型输出:文章被点击的概率
  \end{itemize}

\item \textbf{LightGBM优势}:
  \begin{itemize}
  \item \textbf{直方图算法}:将连续特征离散化,减少内存和计算开销
  \item \textbf{Leaf-wise生长}:只分裂增益最大的叶子节点,相比Level-wise更高效
  \item \textbf{特征并行和数据并行}:支持分布式训练
  \item \textbf{类别特征支持}:原生支持类别特征,无需one-hot编码
  \end{itemize}


\item \textbf{缺点}:无法直接优化排序指标(如NDCG),只能间接通过分类准确率优化
  
\end{enumerate}

\textbf{实际配置:}

\begin{lstlisting}[language=Python, frame=tb, caption=LightGBM配置参数]
lgb.LGBMClassifier(
    num_leaves=64,              # 叶子节点数,控制模型复杂度
    max_depth=10,               # 树的最大深度,防止过拟合
    learning_rate=0.05,         # 学习率,控制每棵树的贡献
    n_estimators=10000,         # 最大迭代次数
    subsample=0.8,              # 行采样比例,防止过拟合
    feature_fraction=0.8,        # 特征采样比例,增加随机性
    reg_alpha=0.5,              # L1正则化
    reg_lambda=0.5,             # L2正则化
    scale_pos_weight=10,        # 正样本权重,处理类别不平衡(1:196)
    min_child_samples=100,      # 叶子节点最小样本数,防止过拟合
)
\end{lstlisting}

\textbf{配置说明:}
\begin{itemize}
\item \textbf{scale\_pos\_weight=10}:由于正负样本比例为1:196,设置正样本权重为10,平衡样本分布
\item \textbf{早停机制}:使用\texttt{early\_stopping(stopping\_rounds=100)},验证集性能不再提升时停止训练
\item \textbf{交叉验证}:使用\texttt{GroupKFold(n\_splits=5)}按用户分组,避免同一用户出现在训练集和验证集
\end{itemize}

\textbf{特征重要性Top 10:}

\begin{table}[H]
\centering
\small
\caption{LightGBM特征重要性Top 10}
\begin{tabular}{clc}
\toprule
\textbf{排名} & \textbf{特征名称} & \textbf{重要性分数} \\
\midrule
1 & sim\_score & 767,792 \\
2 & user\_last\_click\_timestamp\_diff & 237,118 \\
3 & user\_last\_click\_article\_itemcf\_sim & 155,210 \\
4 & user\_last\_click\_article\_w2v\_sim & 91,132 \\
5 & article\_id\_cnt & 79,432 \\
6 & user\_clicked\_article\_itemcf\_sim\_sum & 69,579 \\
7 & user\_last\_click\_created\_at\_ts\_diff & 53,313 \\
8 & user\_last\_click\_article\_binetwork\_sim & 43,189 \\
9 & user\_click\_timestamp\_created\_at\_ts\_diff\_mean & 29,134 \\
10 & user\_click\_article\_w2w\_sim\_sum\_2 & 24,231 \\
\bottomrule
\end{tabular}
\end{table}

\textbf{训练结果(5折交叉验证):}

\begin{table}[H]
\centering
\small
\caption{LightGBM训练结果}
\begin{tabular}{lccccc}
\toprule
\textbf{指标} & \textbf{@5} & \textbf{@10} & \textbf{@20} & \textbf{@40} & \textbf{@50} \\
\midrule
HitRate & 0.46 & 0.59 & 0.70 & 0.77 & 0.78 \\
MRR & 0.281 & 0.298 & 0.306 & 0.309 & 0.309 \\
\bottomrule
\end{tabular}
\end{table}

\textbf{训练时间:} 每个fold约76秒,总计约6.3分钟


\subsubsection{LambdaMART(Listwise)}

\textbf{理论原理:}

LambdaMART是结合LambdaRank和MART的排序算法,属于Listwise方法:

\begin{enumerate}
\item \textbf{LambdaRank核心思想}:
  \begin{itemize}
  \item 定义每个样本的Lambda梯度,直接优化NDCG等排序指标
  \item Lambda值反映交换两个样本位置后NDCG的变化
  \item 对于相关文档,如果排在无关文档后面,Lambda会增大,促使模型提升其分数
  \end{itemize}

\item \textbf{MART实现}:
  \begin{itemize}
  \item 使用梯度提升树拟合Lambda梯度
  \item 每棵树预测Lambda值,多棵树累加得到最终排序分数
  \item 通过梯度提升逐步优化排序性能
  \end{itemize}

\item \textbf{优势}:
  \begin{itemize}
  \item \textbf{直接优化排序指标}:相比Pointwise方法,直接优化NDCG,理论上性能更优
  \item \textbf{考虑列表上下文}:能够利用列表级别的信息(如列表大小、多样性等)
  \item \textbf{梯度提升优势}:继承了GBDT的优点(特征交互、非线性建模)
  \end{itemize}


  \item \textbf{缺点}:实现复杂,需要group信息;训练时间可能较长
  \item \textbf{适用场景}:对排序性能要求高、有列表级别特征的场景
  
\end{enumerate}

\textbf{实际配置:}

\begin{lstlisting}[language=Python, frame=tb, caption=LambdaMART配置参数]
lgb.LGBMRanker(
    num_leaves=128,              # 更大的叶子数,增加模型容量
    max_depth=12,                # 更深的树,捕捉复杂特征交互
    learning_rate=0.03,          # 更小的学习率,更稳定的训练
    n_estimators=15000,          # 更多迭代次数
    subsample=0.85,              # 行采样
    feature_fraction=0.85,       # 特征采样
    reg_alpha=0.2,               # 较小的正则化,允许更复杂的模型
    reg_lambda=0.2,
    objective='lambdarank',      # 使用lambdarank目标函数
    metric='ndcg',               # 评估指标为NDCG
    ndcg_eval_at=[5, 10, 20, 50], # NDCG计算位置
    lambdarank_truncation_level=30, # Lambda计算时的截断级别
)
\end{lstlisting}

\textbf{配置说明:}
\begin{itemize}
\item \textbf{objective='lambdarank'}:使用LambdaRank目标函数,直接优化NDCG
\item \textbf{group信息}:需要提供每个用户的样本数量,用于计算NDCG
\item \textbf{lambdarank\_truncation\_level=30}:只考虑前30个位置的样本对,减少计算量
\item \textbf{更大的模型容量}:相比LGB,使用更多的叶子节点和更深的树,以捕捉复杂的排序模式
\end{itemize}

\textbf{特征重要性Top 10:}

\begin{table}[H]
\centering
\small
\caption{LambdaMART特征重要性Top 10}
\begin{tabular}{clc}
\toprule
\textbf{排名} & \textbf{特征名称} & \textbf{重要性分数} \\
\midrule
1 & sim\_score\_diff\_from\_avg & 952,547 \\
2 & sim\_score & 427,483 \\
3 & user\_last\_click\_timestamp\_diff & 278,883 \\
4 & user\_last\_click\_article\_w2v\_sim & 127,883 \\
5 & article\_id\_cnt & 88,222 \\
6 & list\_rank & 72,591 \\
7 & user\_last\_click\_created\_at\_ts\_diff & 50,636 \\
8 & user\_clicked\_article\_itemcf\_sim\_sum & 46,088 \\
9 & user\_last\_click\_article\_itemcf\_sim & 36,868 \\
10 & user\_last\_click\_article\_binetwork\_sim & 28,220 \\
\bottomrule
\end{tabular}
\end{table}

\textbf{训练结果(5折交叉验证):}

\begin{table}[H]
\centering
\small
\caption{LambdaMART训练结果}
\begin{tabular}{lccccc}
\toprule
\textbf{指标} & \textbf{@5} & \textbf{@10} & \textbf{@20} & \textbf{@40} & \textbf{@50} \\
\midrule
HitRate & 0.46 & 0.59 & 0.70 & 0.77 & 0.78 \\
MRR & 0.282 & 0.300 & 0.307 & 0.310 & 0.310 \\
\bottomrule
\end{tabular}
\end{table}

\textbf{训练时间:} 每个fold约35-40秒,总计约3.2分钟


\textbf{列表级别特征分析:}

LambdaMART能够有效利用列表级别特征,这些特征是Listwise方法独有的,Pointwise方法很难利用。这些特征帮助模型在列表内部进行比较,找到真正相对更好的文章。

\begin{table}[H]
\centering
\small
\caption{列表级别特征分析}
\begin{tabular}{p{6cm}cp{7cm}}
\toprule
\textbf{特征名称} & \textbf{重要性} & \textbf{业务含义} \\
\midrule
sim\_score\_diff\_from\_avg & 952,547 & \textbf{最重要的列表特征}。反映文章在候选列表中的相对位置(相对相似度),而非绝对相似度。因为不同用户的候选列表质量不同,相对位置比绝对位置更重要。让模型能在列表内部比较,找到真正相对更好的文章。 \\
\midrule
list\_rank & 72,591 & 反映召回阶段对文章的置信度。排名越靠前,说明召回算法认为越相关。让排序模型能参考召回阶段的判断。 \\
\midrule
sim\_score\_diff\_from\_max & 24,062 & 反映候选文章与最佳候选的差距。帮助判断文章在列表中的相对质量,差距太大可能不值得推荐。 \\
\bottomrule
\end{tabular}
\end{table}

\subsubsection{模型对比分析}

\textbf{实际性能对比:}

\begin{table}[htbp]
\centering
\caption{LightGBM vs LambdaMART实际性能对比}
\begin{tabular}{lcccc}
\toprule
指标 & LightGBM & LambdaMART & 差异  \\
\midrule
MRR@5 & 0.281 & 0.282 & +0.001  \\
MRR@10 & 0.298 & 0.300 & +0.002  \\
MRR@20 & 0.306 & 0.307 & +0.001  \\
MRR@50 & 0.309 & 0.310 & +0.001 \\
训练时间 & 6.3分钟 & 3.2分钟 & -49\%  \\
\bottomrule
\end{tabular}
\end{table}

\textbf{性能差异分析:}

\begin{enumerate}
\item \textbf{排序性能}:LambdaMART略优于LightGBM(+0.1-0.2\%),符合理论预期
  \begin{itemize}
  \item \textbf{原因分析}:LambdaMART直接优化NDCG,而LightGBM只能间接优化
  \item \textbf{提升幅度小}:可能因为特征工程已经很好,两种方法都能学到有效的排序模式
  \end{itemize}

\item \textbf{训练速度}:LambdaMART反而更快(-49\%),与理论预期不符
  \begin{itemize}
  \item \textbf{可能原因}:
    \begin{itemize}
    \item LambdaMART使用了\texttt{lambdarank\_truncation\_level=30},只计算前30个位置的Lambda,减少了计算量
    \item LightGBM需要计算所有样本的概率,而LambdaMART只需要计算排序分数
    \item 数据分布可能使得LambdaMART的梯度计算更高效
    \end{itemize}
  \end{itemize}

\item \textbf{特征重要性差异}:
  \begin{itemize}
  \item \textbf{LightGBM}:最关注\texttt{sim\_score}(绝对相似度)
  \item \textbf{LambdaMART}:最关注\texttt{sim\_score\_diff\_from\_avg}(相对相似度)
  \item \textbf{结论}:LambdaMART更善于利用列表上下文信息,体现了Listwise方法的优势
  \end{itemize}
\end{enumerate}

\textbf{结论:}
\begin{enumerate}
\item \textbf{性能对比}:LambdaMART在各项指标上略优于LightGBM,符合理论预期(直接优化排序指标)
\item \textbf{训练效率}:LambdaMART训练速度更快,超出理论预期(可能由于截断级别优化)
\item \textbf{特征利用}:LambdaMART更关注列表级别特征(如sim\_score\_diff\_from\_avg、list\_rank),体现了listwise方法的优势
\item \textbf{最终选择}:由于LambdaMART性能略优且训练更快,我们选择LambdaMART作为最终模型
\item \textbf{理论验证}:实际结果验证了Listwise方法在排序任务上的理论优势,虽然提升幅度较小,但在大规模推荐系统中,即使是0.1\%的提升也具有重要价值
\end{enumerate}

\section{训练}

\subsection{训练流程}

训练流程包括数据预处理、召回、特征工程、模型训练和预测五个阶段,总耗时约14-17分钟。

\begin{table}[H]
\centering
\small
\caption{训练流程与时间统计}
\begin{tabular}{p{3cm}cp{6cm}}
\toprule
\textbf{阶段} & \textbf{耗时} & \textbf{主要内容} \\
\midrule
数据预处理 & 约2分钟 & 加载点击日志、划分验证集(50,000用户)、数据清洗和格式转换 \\
\midrule
召回阶段 & 约3分钟 & ItemCF召回(82秒)、Binetwork召回(23秒)、W2V召回(23秒)、召回融合(47秒) \\
\midrule
特征工程 & 约5分钟 & 基础特征、相似度特征、时间特征、Listwise特征(共30个特征) \\
\midrule
模型训练 & 3-6分钟 & LightGBM(5折,6.3分钟)或LambdaMART(5折,3.2分钟) \\
\midrule
预测和提交 & 约1分钟 & 生成预测结果、格式转换和保存 \\
\midrule
\textbf{总计} & \textbf{14-17分钟} & \textbf{离线验证模式} \\
\bottomrule
\end{tabular}
\end{table}

\subsection{关键优化点}

我们在训练过程中采用了多项优化策略,显著提升了训练效率和模型性能:

\begin{table}[H]
\centering
\small
\caption{关键优化策略}
\begin{tabular}{p{3.5cm}p{8cm}}
\toprule
\textbf{优化策略} & \textbf{效果说明} \\
\midrule
向量化计算 & 将相似度特征计算从apply改为向量化,速度提升10-50倍 \\
\midrule
交叉验证 & 使用GroupKFold按用户划分,避免同一用户出现在训练集和验证集,防止数据泄露 \\
\midrule
早停机制 & 设置early\_stopping(stopping\_rounds=100),验证集性能不再提升时停止训练,避免过拟合 \\
\midrule
样本权重 & 使用scale\_pos\_weight=10处理类别不平衡(正负样本比例1:196) \\
\midrule
特征缓存 & 保存特征文件,避免重复计算,提升迭代效率 \\
\bottomrule
\end{tabular}
\end{table}

\section{比赛排名}

成绩 0.2911, 排名第 307,网站图片如下:
\begin{figure}[H]
\centering
\begin{minipage}{0.48\textwidth}
\includegraphics[width=\textwidth]{assets/commit.png}
\end{minipage}
\hfill
\begin{minipage}{0.48\textwidth}
\includegraphics[width=\textwidth]{assets/ranking.png}
\end{minipage}
\caption{比赛排名}
\end{figure}
\section{团队成员分工}

\begin{table}[htbp]
\centering
\caption{团队成员分工}
\begin{tabular}{lc}
\toprule
成员 & 主要职责 \\
\midrule
李丁 & 特征提取、召回算法实现与优化、实验报告撰写 \\
罗浩民 & 排序模型设计与优化、实验报告撰写 \\
\bottomrule
\end{tabular}
\end{table}

\section{个人总结和感悟}

\subsection{技术收获}

\begin{enumerate}
\item \textbf{推荐系统架构}:深入理解了"召回+排序"两阶段推荐系统的设计和实现
\item \textbf{特征工程}:学会了如何从用户行为数据中提取有效特征,特别是时间序列特征和交互特征
\item \textbf{模型对比}:通过对比Pointwise和Listwise方法,理解了不同排序学习范式的优缺点
\item \textbf{性能优化}:掌握了向量化计算、并行处理等优化技巧
\end{enumerate}

\subsection{经验总结}

\begin{enumerate}
\item \textbf{迭代优化的重要性}:通过多次迭代,逐步提升模型性能,每次迭代都有明确的改进目标
\item \textbf{特征工程是关键}:好的特征比复杂的模型更重要,sim\_score等召回特征贡献了最大的重要性
\item \textbf{处理不平衡数据}:类别不平衡是推荐系统的常见问题,需要合理设置样本权重
\item \textbf{计算效率优化}:向量化计算可以大幅提升特征工程效率,值得投入时间优化
\end{enumerate}

\subsection{未来改进方向}

\begin{enumerate}
\item \textbf{深度学习方法}:尝试使用深度神经网络(如DeepFM、Wide\&Deep)进一步提升性能
\item \textbf{实时特征}:增加实时特征,如用户当前会话信息、实时热门文章等
\item \textbf{多目标优化}:不仅优化点击率,还可以考虑多样性、新颖性等指标
\item \textbf{模型融合}:尝试融合多个模型(如LGB + LambdaMART + 深度学习模型)进一步提升性能
\end{enumerate}


\end{document}
